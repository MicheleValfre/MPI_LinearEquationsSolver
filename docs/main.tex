\documentclass[twocolumn]{article}

\usepackage{listings}
\usepackage{color}
\usepackage{amsmath}
\usepackage{algorithm}
\usepackage[noend]{algpseudocode}


\title{Relazione Progetto SCPD}
\date{Ottobre 2022}
\author{Michele Valfrè}
\begin{document}
	\maketitle
	\section{Introduzione: Metodo di Jacobi}
	Il metodo di Jacobi è un metodo iterativo per la soluzione di sistemi di equazioni lineari nella forma $Ax=b$ tali che $A$ sia diagonalmente dominante, ovvero
	\begin{equation}\label{f1}
		\forall i,\ |a_{ii}| >= \sum_{j \neq i} |a_{ij}| 
	\end{equation}
	
	Il passo $k$-esimo dell' iterazione è descritto dall' equazione
	\begin{equation}\label{f2}
		x^{k}_i = \frac{1}{a_{i,i}}[b_i  - \sum_{j\neq i}a_{i,j}x^{k-1}_j].
	\end{equation}
	La complessità del metodo di Jacobi dipende dal numero di iterazioni e dalla precisione richiesta.
	%TODO aggiungere altre info(convergenza, etc...)
	\section{Algoritmo Sequenziale}
	Nell' implementazione sequenziale, si è scelto di utilizzare la seguente classe:
	\lstinputlisting[language=C++,
						  emph={int,char,double,float,unsigned,long},
						  emphstyle={\color{blue}},
						  basicstyle=\ttfamily\scriptsize,
						  columns=fullflexible,
						  linerange={11-17}]
						  {"../src/jacobi.h"}
	
	l' algoritmo utilizzato itera sul vettore $system.x$ delle variabili, per il numero di iterazioni specificate, ed applica ad ognuna delle componenti la formula \eqref{f2}:
	
	\begin{algorithm}
		\footnotesize
		\caption{Jacobi Sequenziale}
		\begin{algorithmic}[1]
			\Function{JACOBI\_SEQ}{system,iterations}
				\State $old\_x \gets system.x$
				\While{$iterations > 0$}
				\State $i \gets 0$
				\While{$i < system.cols$}
				\State $sum \gets \sum_{j\neq i} system.A[i] - old\_x[j]$%COMPUTE\_SUM(system,i,old\_x)$
				\State $old\_x[i] \gets system.x[i]$
				\State $system.x[i] \gets \frac{1}{system.A[i][i]}(system.b[i] - sum)$
				\State $i \gets i + 1$
				\EndWhile
				\State $iterations \gets iterations - 1$
				\EndWhile
			\EndFunction
		\end{algorithmic}
	\end{algorithm}
	\pagebreak
	Il programma riceve tre parametri: tipo di algoritmo(S o P), numero di iterazioni e un file contenente la definizione del sistema in formato csv nel quale l'ultima colonna è considerata come il vettore $b$. Una volta parsato il file, il vettore $x$ viene inizializzato a $0.0$ e, a seconda del tipo di algoritmo specificato, il programma decide se invocare il metodo $jacobi\_seq(...)$ o $jacobi\_par(...)$.
	\section{Algoritmo Parallelo}
	
	\section{Sperimentazione}
	
\end{document}